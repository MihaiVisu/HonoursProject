
% to choose your degree
% please un-comment just one of the following
\documentclass[bsc,frontabs,twoside,singlespacing,parskip,deptreport]{infthesis}     % for BSc, BEng etc.
% \documentclass[minf,frontabs,twoside,singlespacing,parskip,deptreport]{infthesis}  % for MInf

\usepackage{graphicx}
\usepackage{float}

\begin{document}

\title{Sensing Spaces: Personal Exposure to Air Pollution with Different Modes of Transport and Urban Environments Using Wearable Sensors}

\author{Mihai Visuian}

% to choose your course
% please un-comment just one of the following
%\course{Artificial Intelligence and Computer Science}
%\course{Artificial Intelligence and Software Engineering}
%\course{Artificial Intelligence and Mathematics}
%\course{Artificial Intelligence and Psychology }   
%\course{Artificial Intelligence with Psychology }   
%\course{Linguistics and Artificial Intelligence}    
\course{Computer Science}
%\course{Software Engineering}
%\course{Computer Science and Electronics}    
%\course{Electronics and Software Engineering}    
%\course{Computer Science and Management Science}    
%\course{Computer Science and Mathematics}
%\course{Computer Science and Physics}  
%\course{Computer Science and Statistics}    

% to choose your report type
% please un-comment just one of the following
%\project{Undergraduate Dissertation} % CS&E, E&SE, AI&L
%\project{Undergraduate Thesis} % AI%Psy
\project{4th Year Project Report}

\date{\today}

\abstract{
This is an example of {\tt infthesis} style.
The file {\tt skeleton.tex} generates this document and can be 
used to get a ``skeleton'' for your thesis.
The abstract should summarise your report and fit in the space on the 
first page.
%
You may, of course, use any other software to write your report,
as long as you follow the same style. That means: producing a title
page as given here, and including a table of contents and bibliography.
}

\maketitle

\section*{Acknowledgements}

I would like to thank my supervisor, professor D K Arvind for his main support in helping me work thoroughly on a project of such high complexity and importance in the real world. I would also like to thank him for his valuable advice and guidance.

Furthermore, I would like to thank Andrew Bates for his support in providing equipment and helping with maintenance of sensors and servers.

\tableofcontents

%\pagenumbering{arabic}


\chapter{Introduction}

\section{Motivation}

Air pollution in large urban environments is a controversial topic nowadays and pollution sources, as well as distribution and effect on human health have been widely studied. Particulate matter (PM) represents the sum of all solid particles and liquid droplets of small sizes that are suspended in the air. It has been discovered that exposure to such particles of size less than 10 microns can be hazardous for the human body and it has been associated with increased mortality rates in areas labelled as highly polluted \cite{Dockery1994}.

Air quality is mostly measured through stationary Air Quality Monitoring Stations (AQMS). They measure air quality at a fixed location on a certain radius with high precision. Data collected from such networks of stations distributed strategically within urban environments is used to determine exposure to pollutants and is used for checking if a specific area meets requirements set by legislation. One drawback of this approach is that monitoring a large fixed region through a stationary sensor would not always suffice to obtain accurate information about air quality and pollution sources, as parameters vary from a corner of the region to another and other external factors such as wind speed and terrain type affect the results.

One approach to solve this issue and obtain more accurate data from the environment would be to deploy low-cost mobile sensors on people, such as drivers, pedestrians and cyclists in order to obtain a more precise spatial representation of the air quality. However, these sensors have to be calibrated in correlation with the Air Quality Monitoring Stations to ensure high precision of the information gathered.

\section{Objectives}

This project mainly focuses on Mobile Exposure Monitors (MEM) developed by the Centre for Speckled Computing. These wearable devices are equipped with sensors measuring humidity, temperature and particulate matter. Data is transmitted to an Android device via Bluetooth Low Energy. The project presents several study methods for detecting urban environments and personal exposure with different means of transport, based on air quality data collected with the wearable sensors. The objectives of the project are:

\begin{itemize}
\item Collect data at specific times on a daily basis and on a fixed route.
\item Perform analysis on data gathered for various modes of transport and urban environments and research machine learning techniques to detect personal exposure on a specific mode of transport or urban environment based on temperature, humidity and particulate matter attributes.
\item Build a data visualization tool for a more accurate spatial representation of data gathered and results obtained after applying statistical and machine learning methods.
\item Validate measurements of sensors against a more accurate Air Quality Monitoring Station in Edinburgh.
\end{itemize}

\section{Literature Review}

There are past studies concerning exposure of cyclists to air pollution and traffic noise in central neighbourhoods of Montreal \cite{Apparicio201663}. Despite it is known that cycling in urban areas would have beneficial effects on people's health, it is associated with potential health risks due to long time exposure to polluted air and traffic noise. The focus of this study was to analyse exposure of cyclists to air pollution and traffic noise and detect the impact on exposure of local factors such as weather conditions and the type of roads. A number of 85 bicycle trips were analysed, totalling 422 km of travel. The results revealed a weak negative correlation between noise and air pollution (NO$_2$) measures of exposure.

A study \cite{Dirks2012} done in Auckland, New Zealand, examined personal exposure to air pollution, namely carbon monoxide (CO) for several modes of transport, using Langan T15n \cite{Langan} portable carbon monoxide monitors. Participants were constrained to travel at specific peak traffic times between designated start and end points. Results suggest that lowest exposure to CO particles are experienced by train commuters, while motorcyclists were exposed to significantly higher average concentrations. Furthermore, travel by bus on a dedicated route proved to be more effective than travel by car on a congested motorway. Also, average exposure of cyclists and pedestrians proved to be similar to bus and car commuters. However, when increased physical activity that implies higher volumes of air breathed, along with increased commuting time were taken into account, air pollution dosage became much higher than for the motorised means of transport.

Most research regarding particulate matter has been performed in terms of values of PM2.5 and PM10, namely measurements of mass particulates with lower dimensions than 2.5 and 10 microns. A study \cite{Cho2008} is concerned about the relationship between mortality rate and particulate matter measured in Seoul, South Korea by an Optical Particulate Counter (OPC) and a national monitoring station. Particulate matter (PM2.5 and PM10) and particle counts from 0.3 to 25 microns were measured. The results concerning particle counts were associated with a 5.73\% and 5.82\% in mortality caused by respiratory diseases.

\section{Method}

Data collection has been performed in several phases, using a monitoring device worn around the chest, measuring particulate matter (PM1, PM2.5 and PM10 values along with particle counts of different sizes), as well as temperature and humidity. The sensor communicates with an Android device via Bluetooth to which the data gathered is sent. First, data was gathered on Nicholson Street while using a taxi to Calton Hill during midday. Then, data was gathered while walking on the same route, at the same time of the day, in order to perform the first classification analysis between different means of transport (walking as opposed to cars). For further investigation of personal exposure with different modes of transport, another set of data collected was obtained from a train trip from Edinburgh to London King's Cross, including a bus trip to Edinburgh Waverley, as well as walking trips in Edinburgh, London and London Underground.

Data analysis has been performed using Python and Jupyter Notebooks. First, the raw data was labelled accordingly in terms of the mean of transport used at each point in time. Secondly, the raw data was checked for any outliers existing in all the values of interest. After removing all irrelevant outliers, several machine learning classifiers have been used on PM data, along with temperature and humidity, in order to perform a classification of means of transport used during personal exposure experiments.

As far as personal exposure with different urban environments is concerned, a data collection schedule has been organised on a route in Edinburgh, including Nicholson Street, Melville Drive, Middle Meadow Walk, Tollcross and Lauriston Place, so that all types of urban environments would be detected. More exactly, the same route was covered over a week at two different times of the day, first at midday, during lunchtime between 13:00 and 14:00 and then in the evening, between 17:00 and 18:00, when a traffic peak hour is most probable to occur.

As in the previous set of experiments, Jupyter Notebooks have been used for analysis of the dataset regarding different urban environments. In this case, unsupervised machine learning methods have been applied for classification of data, as ranges of PM values corresponding to each urban environment type vary from a region to another. Thus, K-means clustering was performed first on the coordinates of the dataset features, such that all samples would be grouped into compact clusters and the mean values of the PM related values would be computed for each cluster. Then, a second K-means clustering was applied to the mean values of the clusters obtained in the previous step in order to detect the different types of urban environments.

\section{Structure of the Report}

Chapter 2 reveals detailed information regarding equipments used in the data collection phase, as well as details about PM measurements obtained. Moreover, details about the modes of transport used and types of urban environments taken into consideration are presented.

Chapter 3 details the data collection phase. Then, it also introduces details about statistics and machine learning techniques used in the data analysis process. Moreover, it introduces a data visualization tool implemented from scratch to ease analysis of data collected. Details about this tool will be further discussed in the following chapter.

Chapter 4 details on the data visualization tool, focusing both on the purpose and the functionality of the tool. The application is formed of two main parts, namely the front-end dealing with the spatial representation of the data on a map and the back-end, which consists of an API making requests to the data collected which is stored in a database.

Chapter 5 presents the results obtained from the data analysis phase. First, a research regarding classification of pollution exposure with different modes of transport is done using various machine learning classifiers. Then, unsupervised data is used to detect several types of urban environments, which will be presented in detail in chapter 2.

Finally, chapter 6 presents the conclusions drawn from the results obtained from data analysis and described in detail in chapter 5. Moreover, potential improvements and further study areas regarding the main topic of the project worth exploring are described.

\chapter{State of the Art}

\section{Mobile Exposure Monitor}

The Mobile Exposure Monitor (MEM) is a wearable device developed with the purpose of quality of environment measurement by the Centre for Speckled Computing. It is equipped with sensors measuring particulate matter counts and mass (Alphasense OPC-N1), as well as temperature and humidity (Sensirion SHT-75). The device transmits sensor readings via Bluetooth Low Energy (BLE) technology to a smart phone or tablet.

\section{Alphasense Optical Particulate Counter}

\section{Urban Environments}

As far as urban environments detection is concerned, five different types of urban environments are taken into consideration in terms of the average values of the bin counts:

\begin{itemize}
\item park paths
\item pedestrian areas
\item quiet roads
\item medium traffic areas
\item crowded / high traffic areas
\end{itemize}
\label{list:urban-environments}

\section{Personal Exposure with Modes of Transport}

In order to study personal exposure in all cases, six different exposure situations corresponding to six different modes of transport used have been taken into consideration especially during the data collection process:

\begin{itemize}
\item pedestrian data
\item car
\item train
\item bicycle
\item bus
\item subway (in London)
\end{itemize}

\chapter{Methodology}

\section{Data Collection Process}

To begin with, the Mobile Exposure Monitor transmits sensor readings to the Android device through the Airspeck app, via Bluetooth Low Energy. Values obtained from OPC measurements are sent every 25 seconds along with GPS coordinates. This ensures an effective spatial representation of the created datasets especially when data is collected at walking speed.

During the project development, an annotated dataset with modes of transport used at each point was created for data analysis. Each data point contains information about the temperature, humidity, GPS coordinates and accuracy, along with PM1, PM2.5, PM10 values and particulate matter counts split into 16 bins. Most of the data was collected in the first semester, more exactly during November and December. First data collection was performed during a train journey from Edinburgh to Manchester, on a route between several train stations for one hour.

Moreover, a major phase of data collection involving the analysis of different types of urban environments occurred between the 3rd and the 8th of December. More precisely, data capturing particulate matter counts had been collected using the MEM, by walking on a fixed route around meadows based on a daily schedule for a week (Figure \ref{fig:december_route}). The route was scanned twice a day, both during lunch time between 13:00 and 14:00 and in the afternoon between 17:30 and 18:30, so that the same path would be examined at different times of the day involving different intensities of the traffic.

\begin{figure}[h]
  \center
  \includegraphics[width=\columnwidth]{december_route.png} 
  \caption{This map of Meadows region emphasises the route that was scanned while walking in December based on the schedule mentioned.}
  \label{fig:december_route}
\end{figure}

More data gathered and annotated with the purpose of studying personal exposure with different modes of transport has been obtained from Mark Miller who commuted to work by car and wore the device for two consecutive days. He also used the sensors on a train journey to London and around the city centre, involving bus and subway transport. Hence, such a data addition helped balance the categories in the final dataset representing the modes of transport used.

Moreover, more data was collected by bus several times, on a similar route as the one scanned by Mark Miller (Figure \ref{fig:bus_route}) in order to balance the final dataset categories by adding new bus tagged data points.

\begin{figure}[h]
  \center
  \includegraphics[width=\columnwidth]{bus_route.png} 
  \caption{This map emphasises the route that was scanned while taking the bus several times from Waverley station towards Princes Street, George Street and Telford Road.}
  \label{fig:bus_route}
\end{figure}

Furthermore, Aart Meijer, an MSc student whose honours project involved air quality prediction using wearable sensors on cyclists collected data on a fixed route around Meadows when he was commuting to university every day by bicycle. A part of the dataset containing approximately 1000 data points has been extracted and compared to a new set of data points collected in the evening of the 26th of February 2018, during the beginning of the snow storm, in order to analyse how different weather affects sensor readings of particulate matter related information.



\section{Data Analysis}
\label{sec:data-analysis}

\subsection{Mean of Bin Counts}
\label{subsec:bin-count-means}

First, all particulate matter counts have been taken into consideration for analysis in order to visualise the mean values for each bin count over the entire final dataset. Figure \ref{fig:mean_bins_all} shows the distribution of the means of the values for all bin counts over the entire dataset. As it could be observed, bin 0 has the highest values and hence, the distribution of the particulate matter counts is dominated by particulates in bin 0 (range of particulates from 0.38 to 0.52 microns), followed by bins 1 and 2. Therefore, the initial data analysis and visualisation were emphasised mostly on first three bin counts.

\begin{figure}[H]
  \center
  \includegraphics[width=\columnwidth]{mean_bins_all.pdf}
  \caption{This diagram presents the mean values of all bin counts over the entire final dataset.}
  \label{fig:mean_bins_all}
\end{figure}

\subsection{Bin Count Means on Different Weather Conditions}

Raw data is stored in CSV format and analysis is performed using Python 3.6 and Jupyter Notebooks. In order to check how the weather conditions affect the bin counts, the mean count values of all particulates in the size-resolved bins from 0 to 15 have been calculated for two different datasets collected while using a bicycle on similar routes, but at different times and on different weather conditions. The first dataset was collected by Aart Meijer during the summer of 2015, while the second one was collected on the 26th of February 2018, during the snow storm. Figure \ref{fig:bins_weather} displays the distribution of the bin counts for different weather conditions as previously mentioned. The means of the bin counts have been normalised as described in \ref{subsec:bin-count-normalisation} in order to visualise how the pattern, namely the percentage of each bin value is affected by weather changes.

\begin{figure}[H]
  \center
  \includegraphics[width=\columnwidth]{bins_weather.pdf} 
  \caption{This set of diagrams represents patterns formed by the distribution of the bin counts expressed in percentages for different weather conditions when a bicycle was the mean of transport used.}
  \label{fig:bins_weather}
\end{figure}

As it could be observed, the percentage of bin 0 decreased during winter while the average percentages of the other bins increased compared to the values obtained during summer. However, similarities in the patterns could be observed as the order of the percentage values for all bin counts is the same.

\subsection{Time Series Analysis of PM Data}

Time series analysis has been performed on the dataset obtained when walking around Meadows area (Figure \ref{fig:december_route}) in order to study the variance of the particulate matter counts in time while changing urban environments. First the dataset has been sorted in ascending order by the time of the day and it has been split in two halves. The first half contains the data collected at lunch time, while the second one consists of data gathered in the evening.

Same analysis has also been performed on the dataset from the journey to and around London in order to analyse how the particulate matter values change in time while switching from a mode of transport to another.

As to the counts of the particulate matter, bin 0 has the highest variance, followed by the next two bins, so the first three bins have been taken into consideration for this analysis. Moreover, exponential moving average (EMA) , which is implemented in the scikit-learn Python library, has been used for visualisation of bin counts (only first three bins as they are the most prominent ones), as well as PM1, PM2.5, and PM10 values respectively in order to ease event detection (either change of urban environment or mode of transport) by analysing the means of the attributes at each point. Figures \ref{fig:time_series_london} and \ref{fig:time_series_meadows} present the time series analysis of bin counts and PM values for the London journey and Meadows datasets respectively. Moreover, for the London journey dataset, the normalised bin counts have also been taken into consideration in order to discover whether normalisation would ease the detection of modes of transport for the time series analysis. The vertical dashed lines represent the moment when the use of one mean of transport ended during data collection and a new one began until the occurrence of a new line.

\begin{figure}[H]
  \center
  \includegraphics[width=\columnwidth]{time_series_london.pdf} 
  \caption{This set of diagrams represents the time series analysis of the first three bin counts and PM values for the dataset collected on the journey to London. The Exponential Moving Average is used to express the mean value at each point in time.}
  \label{fig:time_series_london}
\end{figure}

As described in \ref{subsec:bin-count-means}, the first 3 bin counts are the most dominant ones, thus these values were used in the time series analysis and visualisation of particulate matter counts. The first diagram of Figure \ref{fig:time_series_london} presents the exponential moving average of the raw values of the bin counts over time. There could be observed several pollution spikes in the absolute value of bin 0, however no detectable pattern emphasising visible changes of the counts in the event of switching means of transport could be seen.

The next diagram displays the normalised values of the bin counts. In this case, the results have improved, as the percentage level differs visibly form a mean of transport to another, especially for the values of bin 0.

As it could be observed in the third diagram of Figure \ref{fig:time_series_london}, the PM10 values are the most dominant ones across the entire journey. There are pollution spikes from time to time, however, the event of change of a mean of transport is not visually detectable and neither a detectable pattern exists nor a visible difference in the absolute values between different means of transport.

\begin{figure}[H]
  \center
  \includegraphics[width=\columnwidth]{time_series_meadows.pdf} 
  \caption{This set of diagrams represents the time series analysis of the first three bin counts and PM values for the dataset collected while walking in Meadows. The Exponential Moving Average is used to express the mean value at each point in time.}
  \label{fig:time_series_meadows}
\end{figure}

As it could be observed in Figure \ref{fig:time_series_meadows}, there are more frequent spikes in the afternoon dataset than in midday, meaning that traffic intensity was on average higher at that time of the day. Moreover, the bin values time series diagrams show longer intervals of time when the values are high, after which they suddenly decrease and stay at a certain level both in the afternoon and during midday. For instance, between 17:17:00 and 17:22:00, the average values of bin 0 are much higher than the values between 17:22:00 and 17:37:00. After that point, they begin to slowly increase again, which suggest that a change in the urban environment is occurring. As to the values of PM1, PM2.5 and PM10, the latter has the most dominant values, however no detectable change of the pattern in time could be observed, especially in the dataset containing data collected during midday, where most of the data points present values between 0 and 20 with a spike at about 13:07:00.

\subsection{Pattern Comparison for Different Modes of Transport}

The relative values between all 16 bin counts have also been taken into consideration. Thus, the means for all bin counts corresponding to each mode of transport in particular have been calculated over the entire final dataset and plotted in order to observe any differences in the pattern between two or more modes of transport. Moreover, the same analysis has been performed after applying the bin counts normalisation explained in \ref{subsec:bin-count-normalisation}, in order to visualise the percentage of each bin count that affects personal exposure with each mode of transport (Figure \ref{fig:norm-bins}).

\begin{figure}[H]
  \center
  \includegraphics[width=\columnwidth]{norm_bins.pdf} 
  \caption{This set of diagrams represents patterns formed by the distribution of the bin counts expressed in percentages for each mode of transport.}
  \label{fig:norm-bins}
\end{figure}

As it could be observed in Figure \ref{fig:norm-bins}, after normalising the values of the particulate matter counts by using the method described in \ref{subsec:bin-count-normalisation}, the patterns between specific modes of transport are visibly distinguishable. For instance, bins 4 and 5 have much smaller values in the case of the car data compared to walking and bicycle datasets. Moreover, bins 7, 8 and 9 have higher values in the case of the bus data compared to any other mean of transport.

\subsection{Spatial Analysis of PM Data}

As GPS information is made available by the MEM readings in the raw data, a spatial analysis has also been performed on all datasets in order to ease visualisation of urban environment clusters, as well as modes of transport classification predictions performed by the models. The visualisation is performed through a data visualisation tool described thoroughly in \ref{sec:data-visualisation-tool}. The tool consists of a web interface containing a map which plots the predictions of the data points, as well as a menu allowing the user to tweak custom classification models by choosing between different classification methods and attributes. Moreover, GPS data was used for a more generic approach of detecting urban environments described in \ref{subsec:generic-clustering}.


\section{Unsupervised Learning for Urban Environments Detection}

This section emphasises on the methodology performed for detection of different urban environments based on data analysis results explained in section \ref{sec:data-analysis}. The approach is done using unsupervised machine learning technique on several groups of attributes retrieved from sensor readings of the AirSpeck device.

\subsection{K-means Clustering on Particulate Matter Data}

As a first version of the method, a K-means clustering algorithm is proposed as an approach for automatically detecting different urban environment types based on the distribution of particulate matter values, as well as particulate matter counts. In this way, the values that have a continuous pattern obtained during a certain time interval (Figure \ref{fig:time_series_meadows}) would be clustered in the same group. PM1, PM2.5 and PM10 values are tested and the results obtained are compared to the ones when the bin counts are used instead for the K-means clustering method.

\subsection{Generalising Clustering Approach}
\label{subsec:generic-clustering}

There might be cases in which only the absolute values and counts of particulate matter would not be sufficient for a robust clustering of urban environments. As it could be seen in Figure \ref{fig:time_series_meadows}, there is a significant number of individual spikes in both the PM values and bin counts  from time to time, which would lead to a misclassification of the specific data points. Therefore, in order to generalise the environment type detection, a new approach is proposed with the aim of addressing these edge cases of misclassified points and preserving the same urban environment in a certain area. 

Thus, K-means clustering is applied twice in a row. First, location-based clustering is performed based on the coordinates of the data points. The next step involves calculating the means of the attributes for each location cluster. Then, the new means are further clustered into 5 new clusters corresponding to the different urban environments specified in Section \ref{list:urban-environments}. In this way, the classification would be more continuous, as location groups created after the first clustering would be categorised in the same urban environment. Figure \ref{fig:clustering-twice} presents the pipeline of the unsupervised clustering method performed on the data points.

\begin{figure}[H]
  \center
  \includegraphics[width=\columnwidth]{clustering-twice.pdf}
  \caption{This diagram shows the pipeline of the unsupervised learning method for urban environments detection, applied twice on the dataset.}
  \label{fig:clustering-twice}
\end{figure}

\section{Supervised Learning for Modes of Transport Classification}

This section is concerned about the methodology applied for the classification of different modes of transport by using supervised machine learning technique on annotated datasets, the label of each data point representing the mode of transport used at a certain moment.

\subsection{Comparison of Models Performance}

First, several basic models have been trained using the absolute values of the particulate matter counts, as well as the PM1, PM2.5, PM10 values of several datasets containing balanced data points of two or more modes of transport. Moreover, temperature and humidity have been added to the training process in order to visualise how the change of these two attributes in time affects classification accuracy on the validation sets. The models used include logistic regression classifiers, support vector machine classifiers (SVC), k-nearest neighbours (KNN) and random forest classifiers (RF). All classification methods previously mentioned are implemented in scikit-learn Python library.

After training the models, the accuracy score has been obtained by using K-Fold cross validation with 5 folds, so that the dataset would be split in 5 equally-sized sets, one of them representing the validation set and the other four being used for training each model.

\subsection{Normalisation of Bin Counts}
\label{subsec:bin-count-normalisation}

Absolute values for the particulate matter counts when all 16 bins are taken into consideration differ significantly in different weather conditions or different urban environments. Hence, only taking the absolute values of the particulate matter counts into consideration in training of the model would produce over-fitting of the model on the training dataset.

One first approach taken into consideration in order to address this issue is to normalise the bin counts by dividing each of the 16 counts by the total sum of the counts for each data point. Thus, the newly normalised attributes would emphasise the percentage that each particulate matter count would affect personal exposure to individuals when using a certain mode of transport in general.

\subsection{Usage of Urban Environments}

As an attempt to make the classification even more independent of the environment and thus, more generic and accurate for validation sets highlighting edge cases, the results obtained from the urban environments detection method have been used as additional attributes in the model training process for modes of transport classification. Figure \ref{fig:urban-environments-classification-model} displays the final data processing pipeline applied prior to the training of the model.

\begin{figure}[h]
  \center
  \includegraphics[width=\columnwidth]{urban-environments-classification-model.pdf}
  \caption{This diagram presents the data processing pipeline which is performed prior to model training for modes of transport classification.}
  \label{fig:urban-environments-classification-model}
\end{figure}

\chapter{Implementation}

\section{Data Visualisation Tool}
\label{sec:data-visualisation-tool}

\subsection{Front-end Interface}

\subsection{Back-end Server}

\chapter{Results}

\section{Detection of Urban Environments}

\section{Comparison of Personal Exposure with Different Modes of Transport}

\subsection{Generalisation of Classification Model}

\subsection{Bin Counts Normalisation}

\chapter{Conclusions}

\section{Concluding Remarks}

\section{Future Work}

% use the following and \cite{} as above if you use BibTeX
% otherwise generate bibtem entries

\bibliographystyle{plain}
\bibliography{mybibfile}

\end{document}
