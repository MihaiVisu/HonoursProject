
% to choose your degree
% please un-comment just one of the following
\documentclass[bsc,frontabs,twoside,singlespacing,parskip,deptreport]{infthesis}     % for BSc, BEng etc.
% \documentclass[minf,frontabs,twoside,singlespacing,parskip,deptreport]{infthesis}  % for MInf

\begin{document}

\title{Sensing Spaces: A Study in Mapping Air Pollution in Public Spaces using Personal Exposure Monitors}

\author{Mihai Visuian}

% to choose your course
% please un-comment just one of the following
%\course{Artificial Intelligence and Computer Science}
%\course{Artificial Intelligence and Software Engineering}
%\course{Artificial Intelligence and Mathematics}
%\course{Artificial Intelligence and Psychology }   
%\course{Artificial Intelligence with Psychology }   
%\course{Linguistics and Artificial Intelligence}    
\course{Computer Science}
%\course{Software Engineering}
%\course{Computer Science and Electronics}    
%\course{Electronics and Software Engineering}    
%\course{Computer Science and Management Science}    
%\course{Computer Science and Mathematics}
%\course{Computer Science and Physics}  
%\course{Computer Science and Statistics}    

% to choose your report type
% please un-comment just one of the following
%\project{Undergraduate Dissertation} % CS&E, E&SE, AI&L
%\project{Undergraduate Thesis} % AI%Psy
\project{4th Year Project Report}

\date{\today}

\abstract{
This is an example of {\tt infthesis} style.
The file {\tt skeleton.tex} generates this document and can be 
used to get a ``skeleton'' for your thesis.
The abstract should summarise your report and fit in the space on the 
first page.
%
You may, of course, use any other software to write your report,
as long as you follow the same style. That means: producing a title
page as given here, and including a table of contents and bibliography.
}

\maketitle

\section*{Acknowledgements}
Acknowledgements go here. 

\tableofcontents

%\pagenumbering{arabic}


\chapter{Introduction}

The document structure should include:
\begin{itemize}
\item
The title page  in the format used above.
\item
An optional acknowledgements page.
\item
The table of contents.
\item
The report text divided into chapters as appropriate.
\item
The bibliography.
\end{itemize}

Commands for generating the title page appear in the skeleton file and
are self explanatory.
The file also includes commands to choose your report type (project
report, thesis or dissertation) and degree.
These will be placed in the appropriate place in the title page. 

The default behaviour of the documentclass is to produce documents typeset in
12 point.  Regardless of the formatting system you use, 
it is recommended that you submit your thesis printed (or copied) 
double sided.

The report should be printed single-spaced.
It should be 30 to 60 pages long, and preferably no shorter than 20 pages.
Appendices are in addition to this and you should place detail
here which may be too much or not strictly necessary when reading the relevant section.

\section{Using Sections}

Divide your chapters into sub-parts as appropriate.

\section{Citations}

Note that citations 
(like \cite{P1} or \cite{P2})
can be generated using {\tt BibTeX} or by using the
{\tt thebibliography} environment. This makes sure that the
table of contents includes an entry for the bibliography.
Of course you may use any other method as well.

\section{Options}

There are various documentclass options, see the documentation.  Here we are
using an option ({\tt bsc} or {\tt minf}) to choose the degree type, plus:
\begin{itemize}
\item {\tt frontabs} (recommended) to put the abstract on the front page;
\item {\tt twoside} (recommended) to format for two-sided printing, with
  each chapter starting on a right-hand page;
\item {\tt singlespacing} (required) for single-spaced formating; and
\item {\tt parskip} (a matter of taste) which alters the paragraph formatting so that
paragraphs are separated by a vertical space, and there is no
indentation at the start of each paragraph.
\end{itemize}

\chapter{The Real Thing}

Of course
you may want to use several chapters and much more text than here.

% use the following and \cite{} as above if you use BibTeX
% otherwise generate bibtem entries
\bibliographystyle{plain}
\bibliography{mybibfile}

\end{document}
