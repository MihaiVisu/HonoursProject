
% to choose your degree
% please un-comment just one of the following
\documentclass[bsc,frontabs,twoside,singlespacing,parskip,deptreport]{infthesis}     % for BSc, BEng etc.
% \documentclass[minf,frontabs,twoside,singlespacing,parskip,deptreport]{infthesis}  % for MInf

\begin{document}

\title{Sensing Spaces: Personal Exposure with Different Modes of Transport and Urban Environments Using Wearable Sensors}

\author{Mihai Visuian}

% to choose your course
% please un-comment just one of the following
%\course{Artificial Intelligence and Computer Science}
%\course{Artificial Intelligence and Software Engineering}
%\course{Artificial Intelligence and Mathematics}
%\course{Artificial Intelligence and Psychology }   
%\course{Artificial Intelligence with Psychology }   
%\course{Linguistics and Artificial Intelligence}    
\course{Computer Science}
%\course{Software Engineering}
%\course{Computer Science and Electronics}    
%\course{Electronics and Software Engineering}    
%\course{Computer Science and Management Science}    
%\course{Computer Science and Mathematics}
%\course{Computer Science and Physics}  
%\course{Computer Science and Statistics}    

% to choose your report type
% please un-comment just one of the following
%\project{Undergraduate Dissertation} % CS&E, E&SE, AI&L
%\project{Undergraduate Thesis} % AI%Psy
\project{4th Year Project Report}

\date{\today}

\abstract{
This is an example of {\tt infthesis} style.
The file {\tt skeleton.tex} generates this document and can be 
used to get a ``skeleton'' for your thesis.
The abstract should summarise your report and fit in the space on the 
first page.
%
You may, of course, use any other software to write your report,
as long as you follow the same style. That means: producing a title
page as given here, and including a table of contents and bibliography.
}

\maketitle

\section*{Acknowledgements}

I would like to thank my supervisor, professor D K Arvind for his main support in helping me work thoroughly on a project of such high complexity and importance in the real world. I would also like to thank him for his valuable advice and guidance.

Furthermore, I would like to thank Andrew Bates for his support in providing equipment and helping with maintenance of sensors and servers.

\tableofcontents

%\pagenumbering{arabic}


\chapter{Introduction}

\section{Motivation}

Air pollution in large urban environments is a controversial topic nowadays and pollution sources, as well as distribution and effect on human health have been widely studied. Particulate matter (PM) represents the sum of all solid particles and liquid droplets of small sizes that are suspended in the air. It has been discovered that exposure to such particles of size less than 10 microns can be hazardous for the human body and it has been associated with increased mortality rates in areas labelled as highly polluted.

Air quality is mostly measured through stationary Air Quality Monitoring Stations (AQMS). They measure air quality at a fixed location on a certain radius with high precision. Data collected from such networks of stations distributed strategically within urban environments is used to determine exposure to pollutants and is used for checking if a specific area meets requirements set by legislation. One drawback of this approach is that monitoring a large fixed region through a stationary sensor would not always suffice to obtain accurate information about air quality and pollution sources, as parameters vary from a corner of the region to another and other external factors such as wind speed and terrain type affect the results.

One approach to solve this issue and obtain more accurate data from the environment would be to deploy low-cost mobile sensors on people, such as drivers, pedestrians and cyclists in order to obtain a more precise spatial representation of the air quality. However, these sensors have to be calibrated in correlation with the Air Quality Monitoring Stations to ensure high precision of the information gathered.

\section{Objectives}

This project mainly focuses on Mobile Exposure Monitors (MEM) developed by the Centre for Speckled Computing. These wearable devices are equipped with sensors measuring humidity, temperature and particulate matter. Data is transmitted to an Android device via Bluetooth Low Energy. The project presents several study methods for detecting urban environments and personal exposure with different means of transport, based on air quality data collected with the wearable sensors. The objectives of the project are:

\begin{itemize}
\item Collect data at specific times on a daily basis and on a fixed route.
\item Perform analysis on data gathered for various modes of transport and urban environments and research machine learning techniques to detect personal exposure on a specific mode of transport or urban environment based on temperature, humidity and particulate matter attributes.
\item Build a data visualization tool for a more accurate spatial representation of data gathered and results obtained after applying statistical and machine learning methods.
\item Validate measurements of sensors against a more accurate Air Quality Monitoring Station in Edinburgh.
\end{itemize}

\section{Literature Review}

\section{Method}

\section{Structure of the Report}

\chapter{Background}

\chapter{Methodology}

\chapter{Implementation}

\section{Data Visualisation Tool}

\chapter{Conclusions}

\chapter{Results}

% use the following and \cite{} as above if you use BibTeX
% otherwise generate bibtem entries
\nocite{P1}
\nocite{P2}
\bibliographystyle{plain}
\bibliography{mybibfile}

\end{document}
